\documentclass[a4paper]{article}
\usepackage[italian]{babel}
\usepackage[utf8]{inputenc}
\usepackage{amssymb,amsmath}

\author{Guido De Rosa}

\begin{document}

\title{Calcolo numerico del periodo di oscillazione di un pendolo semplice}

\maketitle

\section{Introduzione}

L'energia totale di un pendolo semplice è pari all'energia potenziale
nel punto di massima altezza:
\begin{equation*}
  E = mgh = mg(l - l\cos{\theta_0}) .
\end{equation*}
Per la conservzione dell'energia, nel corso dell'oscillazione è 
\[
  E = mg(l - l\cos{\theta}) + \frac{1}{2}mv^2 . 
\]
Confrontando le ultime due equazioni segue che
\[
  v = \sqrt{2g(\cos{\theta} - \cos{\theta_0})} ,
\] 
ed esprimendo la velocità $v$ come $l\frac{d\theta}{dt}$ si ottiene facilmente:
\[
  dt = \sqrt{\frac{l}{2g}}\frac{d\theta}{\sqrt{\cos{\theta} - \cos{\theta_0} }} .
\]
Integrare in $\theta$ fra $0$ e $\theta_0$ equivale a considerare un quarto di periodo, 
dunque:
\[
  T = 4 \sqrt{\frac{l}{2g}} \int_{0}^{\theta_0}\frac{d\theta}{\sqrt{\cos{\theta}-\cos{\theta_0}}}
\]

Si vuole confrontare questa espressione con l'approssimazione di oscillatore armonico, 
secondo la quale per piccoli valori di $\theta_0$ è:
\[
  T \simeq 2\pi \sqrt{\frac{l}{g}} ,
\]
il che si riconduce, dal punto di vista del calcolo, a verificare la bontà 
dell'approssimazione
\[
  \frac{\pi}{\sqrt{2}} \simeq \int_{0}^{\theta_0}\frac{d\theta}{\sqrt{\cos{\theta}-\cos{\theta_0}}}
\]
per valori semper più piccoli di $\theta_0$.
\end{document}
