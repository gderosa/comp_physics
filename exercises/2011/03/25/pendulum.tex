\documentclass[a4paper,12pt]{article}
\usepackage[italian]{babel}
\usepackage[utf8]{inputenc}
\author{Guido De Rosa}

\begin{document}

\title{Calcolo numerico del periodo di oscillazione di un pendolo semplice}

\maketitle

\section{Introduzione}

L'energia totale di un pendolo semplice è pari all'energia potenziale
nel punto di massima altezza.

\begin{equation}
  E = mgh = mg(l - l\cos{\theta_0}) \label{eq:aaa} 
\end{equation}

adsdsf \ref{eq:aaa}

\end{document}
